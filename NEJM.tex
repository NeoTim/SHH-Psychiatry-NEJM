\documentclass[a4paper, 12pt]{article}
\usepackage{doi}
\usepackage{geometry}
\usepackage{xeCJK}
\setCJKmainfont{Source Han Serif TW}

\title{精神科作業}
\author{何震邦}
\date{2017 年 10 月 9 日}

\renewcommand{\refname}{參考資料}

\begin{document}
\maketitle
\section{老年憂鬱症}
老年憂鬱症(60 歲以上發病)通常連結著並存的疾病或認知功能障礙。與早年發病的憂
鬱症病人相比,晚年發病者較有可能患有神經學的異常,包括神經心理學測試的缺損以及
神經造影發現的退化都較正常人為高;這些人也具有較高的罹患失智症的風險。這些結果
彰顯器質性疾病可能是老年憂鬱症的重要成因。

共病讓憂鬱症的處置變得更複雜。憂鬱和共病的關係可能是雙向的:生理疾病例如長期疼
痛可能會引發憂鬱,而憂鬱也會讓像是心臟病等疾病的預後變差。此外,共病會導致多重
用藥,可能會發生藥物交互作用或者藥物副作用影響另一疾病。年齡導致藥物代謝率降低
也可能增強藥物的副作用。

憂鬱的老年人有較高的自殺風險。所以憂鬱症的篩檢很重要,不過陽性的篩檢結果需要全
面考慮病人安全才確診並治療憂鬱症。

藥物治療與心理治療都可作為第一線的治療。目前的抗憂鬱劑對於老年人均有效,不過老
年人較易受到藥物副作用的影響。因此選擇性血清素再回收抑制劑 (selective
serotonin reuptake inhibitors, SSRIs) 較適合作為第一線用藥。標準的心理治療方法
對於老年的憂鬱症也有效。

\section{廣泛性焦慮症}
廣泛性焦慮症的特性是長期且持續的煩惱,通常是多焦的(例如同時包含經濟、家庭、健
康、未來)、過量的,且難以控制,通常帶有其他非特異性的心理或生理症狀。此疾病的
英文名 ``generalized anxiety disorder'' 可能會誤導人此為非特異性的焦慮症,而把
其他焦慮症的人歸到此類。

廣泛性焦慮症常併發憂鬱症、酒精或其他藥物濫用、以及生理疾病。在初級照護中,病情
常以其他生理症狀例如頭痛、肌肉緊繃、腸胃道症狀、背痛、失眠等表現。如 GAD-7 等
經驗證的量表可以作為診斷與判斷治療成效的依據。

第一線治療包括認知行為療法,以 SSRI 或血清素與正腎上腺素再回收抑制劑 (serotonin
and norepinephrine reuptake inhibitor, SNRI) 藥物治療,或者合併認知行為療法與一
種前述藥物。Pregabalin 與 buspirone 適合作為第二線或輔助治療。

雖然由於潛在的濫用以及長期副作用,長期給予 benzodiazepines 仍具有爭議,然而此類
藥物仍可在小心監測下開立予其他治療無效的廣泛性焦慮症病人。

\section{失眠}
每週有三分之一的成人對睡眠品質不滿意,包括入睡困難、維持睡眠困難、或太早睡醒。
對多數人來說,這樣的睡眠困難是暫時的,不會造成太大的困擾。不過長期的睡眠障礙通
常與長期壓力或日間功能受損有關。在這種情況下,失眠的診斷是必要的。生活品質、工
傷、曠職,甚至是死亡意外,都和長期失眠有關。失眠也是獨立於憂鬱的自殺嘗試與自殺
死亡的危險因子,獨立於憂鬱。神經心理學測試則顯示失眠會造成複雜認知功能的缺損,
包括工作記憶與注意力轉移,而不僅僅是警覺度降低而已。

\subsection{診斷}
失眠就是持續對睡眠的品質或時間長度不滿,包含至少一項下列症狀:
\begin{itemize}
    \item 入睡困難
    \item 維持睡眠困難,例如常常在睡夢中醒來,或睡夢中醒來後難以再入睡
    \item 過早醒來並且無法再入睡
\end{itemize}

此睡眠障礙造成臨床上顯著的壓力或者日間功能受損,包括疲勞、白天想睡、注意力或記
憶力受損、情緒受到干擾、行為困難,以及職業、學術、人際、社交、或家庭照護功能受
損。

要診斷失眠,此睡眠障礙每週要發生 3 次以上,持續 3 個月以上,並且是在睡眠機會充
足的情況下。

\subsection{治療}
就像許多精神科的疾病一樣,處置方法包括認知行為療法與藥物治療。

\subsubsection{認知行為療法}
\paragraph{縮短睡眠} 縮短躺在床上的時間,固定睡覺時間,以逐漸增進睡眠效率。晚點
上床以增進睡慾。固定睡覺時間以調整晝夜節律。

\paragraph{控制刺激} 避免睡前刺激並促近床舖與睡眠的連結。只在想睡的時候上床,醒
來或夜間焦慮時下床。床舖只用來從事睡眠或性活動。

\paragraph{認知治療} 改掉失眠對日間活動與健康有災難性影響的有害信念。建立對睡
眠正確的期待。回想過去失眠的經驗。

\paragraph{放鬆治療} 在睡眠環境中減少身心壓力。練習放鬆肌肉、呼吸練習或冥想。

\paragraph{睡眠衛生} 減少干擾睡慾或增強興奮的行為。限制咖啡因和酒精的攝取,讓
臥室陰暗且安靜,避免在非睡眠時間小睡,非睡前時間增加運動量,移除臥室內的時鐘。

\subsubsection{藥物治療}
目前沒有證據顯示 benzodiazepines (BZD) 與非 BZD 藥物之間的療效有顯著差異。由
於安眠藥是在睡前服用,開立藥物的策略是依症狀選擇藥物的半衰期。(例如症狀是難以
入睡就選擇短效的藥物,是過早醒來就選擇長效的藥物。)

\nocite{*}
\bibliographystyle{apalike}
\bibliography{NEJM.bib}
\end{document}
